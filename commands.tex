\setlength{\parindent}{0pt} 
\renewcommand\sfdefault{phv}

\makeatletter
    \renewcommand*\l@subsection{\@dottedtocline{2}{0em}{2.3em}}
    \renewcommand*\l@subsection{\@dottedtocline{3}{0em}{3.2em}}
    \renewcommand{\tableofcontents}{%
        \@starttoc{toc}
    }
\makeatother

%\renewcommand{\thesubsection}{\Roman{subsection}}

\newcommand{\pdfimg}[1]{\pdfximage{pics/#1}\pdfrefximage\pdflastximage}
\newcommand{\img}[1]{\mbox{\pdfimg{#1}}}
\newcommand{\imgc}[1]{\begin{center}\img{#1}\end{center}}

\lstdefinelanguage
   [x64]{Assembler}     % add a "x64" dialect of Assembler
   [x86masm]{Assembler} % based on the "x86masm" dialect
   % with these extra keywords:
   {morekeywords={rax,rdx,rcx,rbx,rsi,rdi,rsp,rbp,rip,r8,r9,r10,r11,r12,r13,r14,r15,popq,pushq,movq,subq,addq,xorq,leaq}}
\lstset{
        language=[x64]Assembler,
        numbers=none,
        numberstyle=\tiny,
        numberblanklines=false,
        stepnumber=1,
        numbersep=10pt
}

\newcommand{\cpp}[1]{
    \lstinline[
        language=C++,
        basicstyle=\ttfamily\color{black},
        keywordstyle=\color{blue},
        commentstyle=\color{green},
        stringstyle=\color{red},
    ] {#1}
}
\newcommand{\cppf}[1]{
    \lstinputlisting[
        language=C++,
        basicstyle=\ttfamily\color{black},
        keywordstyle=\color{blue},
        commentstyle=\color{red},
        stringstyle=\color{green},
    ] {code/#1}
}

\newcommand{\asm}[1]{
    \lstinline[
        basicstyle=\ttfamily\color{black},
        keywordstyle=\color{blue},
        commentstyle=\color{red},
        stringstyle=\color{green}
    ] {#1}
}
\newcommand{\asmf}[1]{
    \lstinputlisting[
        numbers=left,
        basicstyle=\ttfamily\color{black},
        keywordstyle=\color{blue},
        commentstyle=\color{red},
        stringstyle=\color{green}
    ] {code/#1}
}

\newcommand{\boostcoroutine}{boost.coroutine \cite{coroutine}\xspace}
\newcommand{\boostfiber}{boost.fiber \cite{fiber}\xspace}

\newcommand{\doc}{N3708 \cite{n3708}\xspace}
\newcommand{\docr}{N3985 \cite{n3985}\xspace}

\newcommand{\tbb}{TBB \cite{tbb}\xspace}

\newcommand{\abschnitt}[1]{
    \addcontentsline{toc}{subsection}{#1}
    \subsection*{#1}
}

\newcommand{\anhang}[1]{
    \addcontentsline{toc}{subsection}{#1}
    \subsection*{#1}
}
