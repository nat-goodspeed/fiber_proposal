\abschnitt{Coroutines}
\subsubsection*{A Quick Recap of the Coroutine Library}
A coroutine is instantiated and called. When the invoker calls a coroutine,
control immediately transfers into that coroutine; when the coroutine yields,
control immediately returns to its caller (or, in the case of symmetric
coroutines, to the designated next coroutine).\\
\newline
A coroutine does not have a conceptual lifespan independent of its invoker.
Calling code instantiates a coroutine, passes control back and forth with it
for some time, and then destroys it. It makes no sense to speak of "detaching"
a coroutine. It makes no sense to speak of "blocking" a coroutine: the
coroutine library provides no scheduler. The coroutine library provides no
facilities for synchronizing coroutines: coroutines are already synchronous.\\
\newline
Coroutines do not resemble threads. A coroutine much more closely resembles an
ordinary function, with a semantic extension: passing control to its caller
with the expectation of being resumed later at exactly the same point. When
the invoker resumes a coroutine, the control transfer is immediate. There is
no intermediary, no agent deciding which coroutine to resume next.\\

\subsubsection*{A Few Coroutine Use Cases}
Normally, when consumer code calls a producer function to obtain a value, the
producer must return that value to the consumer, discarding all its local
state in so doing. A coroutine allows you to write producer code that "pushes"
values (via function call) to a consumer that "pulls" them with a function
call.\\
\newline
For instance, a coroutine can adapt callbacks, as from a SAX parser, to values
explicitly requested by the consumer.\\
\newline
Moreover, the proposed coroutine library provides iterators over a producer
coroutine so that a sequence of values from the producer can be fed directly
into an STL algorithm. This can be used, for example, to flatten a tree
structure.\\
\newline
Coroutines can be chained: a source coroutine can feed values through one or
more filter coroutines before those values are ultimately delivered to
consumer code.\\
\newline
In all the above examples, as in every coroutine usage, the handshake between
producer and consumer is direct and immediate.
