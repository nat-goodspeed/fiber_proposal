\abschnitt{Relationship}
The authors have a reference implementation of the forthcoming fiber library
(\boostfiber). The reference implementation is entirely
coded in portable C++; in fact its original implementation was entirely in
C++03.\\
\newline
This is possible because the reference implementation of the fiber library is
built on \boostcoroutine, which provides context management. The fiber library
extends the coroutine library by adding a scheduler and the aforementioned
synchronization mechanisms.\\
\newline
Of course it would be possible to implement coroutines on top of fibers
instead. But the concepts map more neatly to implementing fibers in terms of
coroutines. The corresponding operations are:

\begin{itemize}
    \item a coroutine yields;
    \item a fiber blocks.
\end{itemize}

When a coroutine yields, it passes control directly to its caller (or, in the
case of symmetric coroutines, a designated other coroutine).\\
\newline
When a fiber blocks, it implicitly passes control to the fiber scheduler.
Coroutines have no scheduler because they need no scheduler.
